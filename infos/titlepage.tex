%---------------------------------------------------------------------------------
%	BOOK INFORMATION
%---------------------------------------------------------------------------------

% 设置标题页顶部的内容,允许用户在标题页上方添加自定义内容,如图片、文本等。
\titlehead{\texttt{kaobook}的中文支持和扩展}
% 设置文档的主题或学科,会作为文档标题页上的一部分,展示为“文档主题”。
\subject{Latex的学习系列内容}
% 设置主标题
\title[Example and documentation of the {\normalfont\texttt{kaobook}} class]{{\normalfont\texttt{kaobook}}文档类的中文适配模板}
% 设置副标题
\subtitle{根据您的需要自定义此页面}
% 作者
\author[Federico Marotta]{Tigcat\thanks{一个 \LaTeX 热爱者}}
% 日期
\date{\today}
% 出版社或公司
\publishers{虚拟书社}

\makeatletter
\uppertitleback{\@titlehead} % Header

\lowertitleback{
    \textbf{免责声明}\\
    你可以编辑此页面以满足你的需求。例如,本文中包含了免责声明、书籍说明及其他一些信息。此页面基于~Ken Arroyo Ohori~的论文相应页面和~\href{https://github.com/fmarotta/kaobook/}{kaobook}~宏包作了最小的修改。
 
    \medskip
 
    \textbf{无版权}\\
    \cczero 基于~\href{https://github.com/fmarotta/kaobook/}{kaobook}~宏包的中文翻译和使用已通过~CC0~许可发布至公共领域。在法律允许的范围内,我放弃对此作品的所有版权及相关或邻接权利。
    
    要查看CC0许可协议,请访问:~\\\url{http://creativecommons.org/publicdomain/zero/1.0/}~
    
    \medskip
    
    \textbf{书籍说明} \\
    本文件使用~\href{https://sourceforge.net/projects/koma-script/}{\KOMAScript}~和~\href{https://www.latex-project.org/}{\LaTeX}~的~\href{https://github.com/fmarotta/kaobook/}{kaobook}~宏包进行排版。
    
    本文件的源代码可在以下地址获取:~\\\url{https://github.com/wenhq/template-kaobook-zh}~(欢迎您进行贡献!)
    
    \medskip
    
    \textbf{出版商} \\
    本文件首次发布于2025年2月,由~\@publishers~出版。
}
\makeatother

\maketitle