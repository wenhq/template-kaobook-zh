\setchapterpreamble[u]{\margintoc}
\chapter{如何使用本项目}

\section{概述}
本项目基于 \href{https://github.com/fmarotta/kaobook}{kaobook 项目}进行中文汉化和二次开发。

kaobook 是一个 LaTeX 文档类,旨在为书籍、报告等长篇文档提供一个优雅、灵活的排版框架。该项目基于 KOMA-Script 构建,并融入了许多独特的设计元素,如宽边距、迷你目录、自定义章节标题等,以满足学术著作和技术文档的高排版要求。

在实际使用 Kaobook 进行文档编写的过程,发现它在处理包含大量不同尺寸图片的文档,尤其是注释说明类操作手册方面表现出色,生成的文档布局整齐、视觉效果良好,且在排版时能够合理分配空间,提升文档的可读性与专业性。

本项目是一个使用 kaobook 的文档模板,为了方便中文文档的快速使用。项目以 kaobook 文档类为基础,通过扩展宏包的方式使用 kao 样式的文档排版,以实现更高效和专业的排版效果。

\section{注意事项}

项目的作者仅为 LaTeX 的爱好者,因此在项目的开发过程中,难免存在一些考虑不周或使用不当的地方。这些问题可能源于作者对 LaTeX 以及 kaobook 特性和功能的理解尚未完全深入,或者由于实践经验的局限性,导致某些细节未能得到充分优化。

尽管该项目能够满足基本需求,但在复杂文档处理、性能优化或特殊排版要求方面,可能会存在不足之处。用户在使用过程中应注意可能遇到的一些局限性,并根据实际需求进行适当的调整和补充。

\section{快速上手}

项目的主文件是 main.tex,按导言区和文档主体两部分进行说明。

\subsection{导言区}

文档类 kaobook 被选为文档的基础类,适用于书籍和长篇文档的排版,支持双面打印(twoside)功能,使得排版更加符合传统书籍的格式要求。为了简化文献引用和参考文献列表的管理,使用了 kaorefs 宏包,它能够自动处理文献的引用格式和生成文献列表,确保文献管理的准确性和一致性。

针对中文排版,kao-zh 宏包提供了优化的中文排版支持,包括字体、行间距、段落以及对kao相关宏包的汉化调整,使得中文文档的排版更加规范和易读。此外,为了满足更复杂的排版需求,kao-ext 宏包被用于提供扩展功能,允许用户根据需要进行更多自定义,如复杂的表格、数学公式和特殊排版效果。

如~\reflistings{文档导言区示例}~所示,使用kaobook中文模板在导言区的宏包引用实例的写法。

\begin{listing}[H]
    \begin{minted}{latex}
        %! Tex program = xelatex
        \documentclass[twoside]{styles/kaobook}
        \usepackage{styles/kaorefs}
        \usepackage{styles/kao-zh}
        \usepackage{styles/kao-ext}
    \end{minted}
    \caption{文档导言区示例}
    \lablistings{文档导言区示例}
\end{listing}

这些宏包的结合,使得本书的排版更加精美、规范,并且方便文献管理,提升了整体的排版质量和效率。

\subsection{文档主体}

在 LaTeX 中, \mintinline{latex}|\frontmatter| 、 \mintinline{latex}|\mainmatter| 和  \mintinline{latex}|\backmatter| 是用于结构化文档内容的重要命令,特别适用于书籍类文档。这些命令帮助区分文档的不同部分,并控制页码格式。具体而言, \mintinline{latex}|\frontmatter| 用于定义文档的前言部分, \mintinline{latex}|\mainmatter| 标志着正文部分的开始,而 \mintinline{latex}|\backmatter| 则用于定义附录等后续内容。

\begin{itemize}
    \item \textbf{\mintinline{latex}|\frontmatter| 命令}标志着文档的前言部分的开始。前言部分通常包含封面、版权信息、致谢、目录等内容,而不涉及文档的正式正文。在这一部分,LaTeX 会使用罗马数字(如 i, ii, iii 等)作为页码格式。这种格式的设置是为了区分前言部分与正文部分,并保持整洁的排版风格。
    \item \textbf{\mintinline{latex}|\mainmatter| 命令}标志着文档正式正文部分的开始。在执行该命令后,文档的页码将从 1 开始,并切换为 阿拉伯数字(如 1, 2, 3 等)\sidenote{ kaobook 文档的 \mintinline{latex}|\mainmatter| 命令不仅仅恢复页码格式,还会影响章节标题的排版,提供更加专业和规范的章节排版样式。同时,正文部分的页面布局也会依据 kaobook 的设计进行调整,通常包括更宽的页边距和专门设计的章节封面。}。这表明正文部分的内容正式开始,通常包括章节、节、小节等内容。
    \item \textbf{\mintinline{latex}|\backmatter| 命令}标志着文档的附录部分的开始。附录部分通常包括参考文献、索引、附加数据等内容。此时,文档的页码继续使用阿拉伯数字,并且不会进行新的章节编号。
\end{itemize}

为了方便管理文档,在文档主体区域的代码写法如~\reflistings{文档主体区域示例}~所示。
\begin{listing}[H]
    \begin{minted}{latex}
        \begin{document}
        %---------------------------------------------------------------------------------
%	BOOK INFORMATION
%---------------------------------------------------------------------------------

% 设置标题页顶部的内容,允许用户在标题页上方添加自定义内容,如图片、文本等。
\titlehead{\texttt{kaobook}的中文支持和扩展}
% 设置文档的主题或学科,会作为文档标题页上的一部分,展示为“文档主题”。
\subject{Latex的学习系列内容}
% 设置主标题
\title[Example and documentation of the {\normalfont\texttt{kaobook}} class]{{\normalfont\texttt{kaobook}}文档类的中文适配模板}
% 设置副标题
\subtitle{根据您的需要自定义此页面}
% 作者
\author[Federico Marotta]{Tigcat\thanks{一个 \LaTeX 热爱者}}
% 日期
\date{\today}
% 出版社或公司
\publishers{虚拟书社}

\makeatletter
\uppertitleback{\@titlehead} % Header

\lowertitleback{
    \textbf{免责声明}\\
    你可以编辑此页面以满足你的需求。例如,本文中包含了免责声明、书籍说明及其他一些信息。此页面基于~Ken Arroyo Ohori~的论文相应页面和~\href{https://github.com/fmarotta/kaobook/}{kaobook}~宏包作了最小的修改。
 
    \medskip
 
    \textbf{无版权}\\
    \cczero 基于~\href{https://github.com/fmarotta/kaobook/}{kaobook}~宏包的中文翻译和使用已通过~CC0~许可发布至公共领域。在法律允许的范围内,我放弃对此作品的所有版权及相关或邻接权利。
    
    要查看CC0许可协议,请访问:~\\\url{http://creativecommons.org/publicdomain/zero/1.0/}~
    
    \medskip
    
    \textbf{书籍说明} \\
    本文件使用~\href{https://sourceforge.net/projects/koma-script/}{\KOMAScript}~和~\href{https://www.latex-project.org/}{\LaTeX}~的~\href{https://github.com/fmarotta/kaobook/}{kaobook}~宏包进行排版。
    
    本文件的源代码可在以下地址获取:~\\\url{https://github.com/wenhq/template-kaobook-zh}~(欢迎您进行贡献!)
    
    \medskip
    
    \textbf{出版商} \\
    本文件首次发布于2025年2月,由~\@publishers~出版。
}
\makeatother

\maketitle
        \frontmatter
        \chapter*{\centering 序言}
\addcontentsline{toc}{chapter}{序言} % Add the preface to the table of contents as a chapter
LaTeX 就像一座桥梁,将数学与艺术、形式与内容、理性与感性完美融合。它不仅展现了世界的和谐与秩序,还赋予了文字与符号生命,使自然哲学的心灵与灵魂得以深刻表达。在它的帮助下,数学之美被提升到一种诗意的境界,每一个公式、每一行文字,都蕴含着无尽的优雅与智慧,体现了人类对知识与美的追求。

\begin{enumerate}
	\item 这段话是瞎写的;
	\item 这段话是用 GPT-4.1 mini 瞎写的。
\end{enumerate}

LaTeX is like a bridge, perfectly blending mathematics with art, form with content, reason with emotion. It not only showcases the harmony and order of the world but also breathes life into words and symbols, allowing the spirit and soul of natural philosophy to be deeply expressed. With its help, the beauty of mathematics is elevated to a poetic realm, where every formula, every line of text, contains infinite elegance and wisdom, reflecting humanity's pursuit of knowledge and beauty.

\begin{enumerate}
	\item This paragraph is nonsense.
	\item This paragraph is nonsense written using GPT-4.1 mini.
\end{enumerate}

\begin{flushright}
	\textit{Tigcat Wh Q}
\end{flushright}

        %----------------------------------------------------------------------------------
%	TABLE OF CONTENTS & LIST OF FIGURES/TABLES
%----------------------------------------------------------------------------------

\begingroup % Local scope for the following commands

% Define the style for the TOC, LOF, and LOT
%\setstretch{1} % Uncomment to modify line spacing in the ToC
%\hypersetup{linkcolor=blue} % Uncomment to set the colour of links in the ToC
\setlength{\textheight}{230\hscale} % Manually adjust the height of the ToC pages

% Turn on compatibility mode for the etoc package
\etocclasstocstyle % "toc display" as if etoc was not loaded
\etocstandardlines % "toc lines" as if etoc was not loaded

\tableofcontents % Output the table of contents

\listoffigures % Output the list of figures

% Comment both of the following lines to have the LOF and the LOT on different pages
\let\cleardoublepage\bigskip
\let\clearpage\bigskip

\listoftables % Output the list of tables

\endgroup

        
        \mainmatter
        \chapter{记录}
在 LaTeX 文档中,\\frontmatter 命令通常用于标记文档的前言部分开始。在 KOMA-Script 文档类(如 scrbook 或 scrreprt)和一些其他文档类中,\\frontmatter 命令之后的内容会使用罗马数字(i, ii, iii, ...)进行页码编号,直到 \\mainmatter 命令,之后的内容会使用阿拉伯数字(1, 2, 3, ...)进行页码编号。

frontmatter 部分通常包括:

封面(Title Page)
版权信息(Copyright Page)
致谢(Acknowledgments)
摘要(Abstract)
目录(Table of Contents)
列表(List of Figures, List of Tables 等)
引言(Introduction)或前言(Preface)
这些部分在书籍和长篇报告中很常见,它们为读者提供了文档的概览和导航。
        \setchapterpreamble[u]{\margintoc}
\chapter{主要变动点}
\labch{主要变动点}

\section[原样式修改]{对原始样式的修改}

\href{https://github.com/fmarotta/kaobook}{kaobook 项目}通过提供一系列 LaTeX 文档类和宏包,为学术著作和技术文档的排版提供了强大的支持。

Kaobook模板相关文件包括 kaobook.cls、kao.sty、kaobiblio.sty、kaorefs.sty 和 kaotheorems.sty,这些文件可以灵活的引用。它们的主要作用为:

\begin{itemize}
    \item \textbf{kaobook.cls‌} 作为主文档类文件定义了文档的整体排版框架,包括页边距、章节样式、多级标题等。它支持分页控制、多语言兼容等书籍排版特性,为文档提供统一且专业的外观。
    \item \textbf{kao.sty} 基础样式包提供了颜色方案、自定义命令和工具函数等底层支持。这些功能被广泛应用于文档中的代码高亮、侧边注释等场景,是其他扩展样式包的基础依赖。
    \item \textbf{kaobiblio.sty} 是参考文献处理模块, 集成并扩展了 biblatex 的功能,预设了符合 kaobook 风格的引用格式(如作者-年份或数字标号)。它支持多文献库管理,简化了参考文献的插入和格式化过程。
    \item \textbf{kaorefs.sty} 交叉引用增强工具优化了图表、章节等元素的智能引用显示。它支持超链接跳转和动态标签生成,提高了文档中交叉引用的可读性和易用性。
    \item 数学环境定制包,\textbf{kaotheorems.sty} 提供了统一风格的定理、引理、证明等数学环境框。这些框支持可折叠功能,便于读者浏览和隐藏数学证明等细节内容。同时,它还支持自定义编号规则和边注标记,满足了复杂数学文档的排版需求。
\end{itemize}

本项目引入了 kaobook.cls、kao.sty和kaorefs.sty三个文件。在引入原始 .cls 或 .sty 文件时,尽量避免修改原文件内容,以便在原项目升级时能够方便地进行集成。尽管如此,仍然不可避免地会遇到一些不适应中文排版的警告或错误,需要针对这些问题进行修复和调整。


\subsection{kaobook.cls‌的变动}
由于要引入 styles 文件夹下的对应文件,因此在第28行,将 \mintinline{latex}|\ProvidesClass{kaobook}| 改为 \mintinline{latex}|\ProvidesClass{styles/kaobook}|;在第44行,将 \mintinline{latex}|\RequirePackage{kao}| 改为 \mintinline{latex}|\RequirePackage{styles/kao}|。

另外,因为中文无法支持“小型大写字母”这样的字体,为避免编译告警故删除了第254和255行的 \mintinline{latex}|\scshape| 命令,如~\reflistings{kaobookcls删除scshape命令}~所示。

\begin{listing}[H]
    \begin{minted}[linenos=false]{latex}
        \addtokomafont{part}{\normalfont\bfseries}
        \addtokomafont{partentry}{\normalfont\bfseries}
    \end{minted}
    \caption{删除 scshape 后的 kaobook.cls‌ 文件}
    \lablistings{kaobookcls删除scshape命令}
\end{listing}

\subsection{kao.sty的变动}
kao.sty文件同样要指定到 styles 文件夹下,修改文件第1行为 \mintinline{latex}|\ProvidesPackage{styles/kao}|。

一些兼容性的修改,如第19行删除了 usenames,改为 \newline \mintinline{latex}|\RequirePackage[dvipsnames,table]{xcolor}|; \newline 第29行增加了 listings=false, 改为 \newline \mintinline{latex}|\AtEndPreamble{\RequirePackage[listings=false]{scrhack}}|; 第240行增加了 singlespacing=true,改为 \newline \mintinline{latex}|\RequirePackage[singlespacing=true]{scrlayer-scrpage}|。

调整了目录的深度,从 section 调整到 subsection,文件第1208行改为 \mintinline{latex}|bookmarksdepth=subsection,|。

\subsection{kaorefs.sty的变动}
kaorefs.sty文件类似地修改到 styles 文件夹下,修改文件第1行为 \mintinline{latex}|\ProvidesPackage{styles/kaorefs}|。

注释掉第41行,以及从48行到148行这段多语言支持的部分。

\section[kao-zh.sty中文支持]{新增kao-zh.sty中文支持文件}

新增的 kao-zh.sty 宏包文件是通过 ctex 包来支持中文的显示,并且通过这个文件对 kaobook 中的关键字展示采用中文重定义,并规定了文档的字体。

\subsection{ctex宏包}

在 kao-zh.sty 文件中引入 ctex 宏包来支持中文,引用时不指定具体字体,后续使用自定义的开源字体代替。

在 \reflistings{kao-zh.sty中引入ctex宏包} 里,如果不准备自定义字体,可以将fontset=none改为对应的mac或windows。

\begin{listing}[H]
    \begin{minted}[linenos=false]{latex}
        \RequirePackage[UTF8, fontset=none, scheme=chinese]{ctex}
    \end{minted}
    \caption{kao-zh.sty中引入ctex宏包}
    \lablistings{kao-zh.sty中引入ctex宏包}
\end{listing}

ctex 是一个专门为支持中文排版而设计的 LaTeX 宏包\sidenote{关于 ctex 宏包的详细文档,可以参考\href{https://www.ctan.org/pkg/ctex}{CTeX 官方文档}},它是LaTeX 中是中文排版里最常用的标准方案,为 LaTeX 提供了对中文字符、字体、段落、编码等的良好支持。

\subsection{标题文本的中文重定义}
对于 kao 样式里定义的一些标题文本,在本项目中进行了重新定义\sidenote{在英文环境中更改特定标签(如目录、插图、表格等)的名称,\textbf{应该有更好的修改方式}。}。如 \reflistings{kao-zh.sty中的标题文本重定义} 里的重新定义命令,可以修改以适配自己需要的中文。

\begin{listing}[H]
    \begin{minted}[linenos=false]{latex}
        \RequirePackage[english]{babel}
        \renewcaptionname{english}{\contentsname}{目录}
        \renewcaptionname{english}{\listfigurename}{插图}
        \renewcaptionname{english}{\listtablename}{表格}
        \renewcaptionname{english}{\indexname}{索引}
        \renewcaptionname{english}{\bibname}{参考文献}
        \renewcaptionname{english}{\figurename}{图}
        \renewcaptionname{english}{\tablename}{表}
        
        \renewcommand{\lstlistlistingname}{代码列表}
    \end{minted}
    \caption{kao-zh.sty中的标题文本重定义}
    \lablistings{kao-zh.sty中的标题文本重定义}
\end{listing}

\subsection{自定义字体}
为了避免使用系统默认字体可能带了的商业使用法律问题,本项目采用了开源免费商用字体。这部分内容可以根据需要自行修改。

\subsubsection{中文字体}

本项目采用 \href{https://github.com/takushun-wu/WenYuanFonts}{WenYuanFonts 项目}提供的文源字体\sidenote{基于思源字体二次开发,开源免费商用。}作为文档字体,另采用 \href{https://github.com/subframe7536/maple-font}{maple-font 项目}提供的Maple Mono 字体\sidenote{开源等宽字体,中英文宽度完美 2:1。}作为文档的等宽字体。所有使用到的字体文件参考 \nrefsec{代码目录结构} 的内容都放在了 font 目录下,且注意,\textbf{字体文件不保证为最新}。

\begin{listing}[H]
    \begin{minted}[linenos=false]{latex}
        \setCJKmainfont [Path={font/}, AutoFakeBold, AutoFakeSlant] {WenYuanMincho-Regular.ttf}
        % 中文无衬线字体
        \setCJKsansfont [Path={font/}, BoldFont={WenYuanGothic-Medium.ttf}, AutoFakeSlant] {WenYuanGothic-Regular.ttf}
        % 中文等宽字体
        \setCJKmonofont [Path={font/}, BoldFont={MapleMonoNormalNL-NF-CN-Bold.ttf},  ItalicFont={MapleMonoNormalNL-NF-CN-Italic.ttf}] {MapleMonoNormalNL-NF-CN-Light.ttf}
    \end{minted}
    \caption{kao-zh.sty中使用的中文字体}
    \lablistings{kao-zh.sty中使用的中文字体}
\end{listing}

如 \reflistings{kao-zh.sty中使用的中文字体} 所示\sidenote{有些地方配置了 AutoFakeBold 和 AutoFakeSlant,没有什么原因,只是懒得去下载字体了。},中文有衬线字体使用文源宋体(WenYuanMincho)、中文无衬线字体使用文源黑体(WenYuanGothic)、等宽字体使用 Maple Mono 字体。

\subsubsection{英文字体}

当前使用了 TeX Gyre 系列的字体,准备改为 Times 字体。\todo{使用Times New Roman字体的正确姿势} 

\iffalse
\section{后续内容}
\subsection{富强民主}
\zhlipsum[3]
\fi
        
        \backmatter
        %----------------------------------------------------------------------------------
%	ACKNOWLEDGMENTS PAGE
%----------------------------------------------------------------------------------
\chapter*{\centering 致谢} \todo{致谢部分待补充}
\addcontentsline{toc}{chapter}{致谢} % 将致谢添加到目录中

本文源于2024年从~\href{https://www.latexstudio.net/}{\LaTeX 工作室} 看到了有关 ~\\\url{https://github.com/JimRou/template_kaobook} 的介绍。
        \end{document}
    \end{minted}
    \caption{文档主体区域示例}
    \lablistings{文档主体区域示例}
\end{listing}

\section{代码目录结构}
\labsec{代码目录结构}
\marginnote{
    \dirtree{%
      .1 template-kaobook-zh.
      .2 \textcolor{cyan}{chapters}.
      .3 ch1.tex.
      .3 ch2.tex.
      .2 \textcolor{cyan}{font}.
      .3 \textcolor{green}{[一系列字体文件]}.
      .2 \textcolor{cyan}{infos}.
      .3 titlepage.tex.
      .3 toc.tex.
      .2 \textcolor{cyan}{styles}.
      .3 kao-ext.sty.
      .3 kao-zh.sty.
      .3 kao.sty.
      .3 kaobook.cls.
      .3 kaorefs.sty.
      .2 main.tex.
      .2 main.pdf.
    }
}
与文档结构对应的是代码目录的结构。在本项目中,将所有要使用的文件组织成若干子目录,确保代码结构清晰、模块化,便于管理与修改。主要的目录有 styles、 infos、 chapters等,说明如下:

\begin{itemize}
  \item \texttt{chapters/}:存放文档的各个章节文件,每个章节的内容分别保存在独立的 \texttt{.tex} 文件中。
  \item \texttt{font/}:用于存放文档所需的字体文件,确保文档的排版符合要求。
  \item \texttt{infos/}:存放文档的信息文件,如封面 (\texttt{titlepage.tex}) 和目录 (\texttt{toc.tex})。
  \item \texttt{styles/}:存放样式文件和类文件,这些文件用于定义文档的样式和格式。
  \item \texttt{main.tex}:文档的主文件,包含所有内容和结构的汇总。
  \item \texttt{main.pdf}:生成的最终 PDF 文件。
\end{itemize}
