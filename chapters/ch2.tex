\setchapterpreamble[u]{\margintoc}
\chapter{主要变动点}
\labch{主要变动点}

\section[原样式修改]{对原始样式的修改}

\href{https://github.com/fmarotta/kaobook}{kaobook 项目}通过提供一系列 LaTeX 文档类和宏包,为学术著作和技术文档的排版提供了强大的支持。

Kaobook模板相关文件包括 kaobook.cls、kao.sty、kaobiblio.sty、kaorefs.sty 和 kaotheorems.sty,这些文件可以灵活的引用。它们的主要作用为:

\begin{itemize}
    \item \textbf{kaobook.cls‌} 作为主文档类文件定义了文档的整体排版框架,包括页边距、章节样式、多级标题等。它支持分页控制、多语言兼容等书籍排版特性,为文档提供统一且专业的外观。
    \item \textbf{kao.sty} 基础样式包提供了颜色方案、自定义命令和工具函数等底层支持。这些功能被广泛应用于文档中的代码高亮、侧边注释等场景,是其他扩展样式包的基础依赖。
    \item \textbf{kaobiblio.sty} 是参考文献处理模块, 集成并扩展了 biblatex 的功能,预设了符合 kaobook 风格的引用格式(如作者-年份或数字标号)。它支持多文献库管理,简化了参考文献的插入和格式化过程。
    \item \textbf{kaorefs.sty} 交叉引用增强工具优化了图表、章节等元素的智能引用显示。它支持超链接跳转和动态标签生成,提高了文档中交叉引用的可读性和易用性。
    \item 数学环境定制包,\textbf{kaotheorems.sty} 提供了统一风格的定理、引理、证明等数学环境框。这些框支持可折叠功能,便于读者浏览和隐藏数学证明等细节内容。同时,它还支持自定义编号规则和边注标记,满足了复杂数学文档的排版需求。
\end{itemize}

本项目引入了 kaobook.cls、kao.sty和kaorefs.sty三个文件。在引入原始 .cls 或 .sty 文件时,尽量避免修改原文件内容,以便在原项目升级时能够方便地进行集成。尽管如此,仍然不可避免地会遇到一些不适应中文排版的警告或错误,需要针对这些问题进行修复和调整。


\subsection{kaobook.cls‌的变动}
由于要引入 styles 文件夹下的对应文件,因此在第28行,将 \mintinline{latex}|\ProvidesClass{kaobook}| 改为 \mintinline{latex}|\ProvidesClass{styles/kaobook}|;在第44行,将 \mintinline{latex}|\RequirePackage{kao}| 改为 \mintinline{latex}|\RequirePackage{styles/kao}|。

另外,因为中文无法支持“小型大写字母”这样的字体,为避免编译告警故删除了第254和255行的 \mintinline{latex}|\scshape| 命令,如~\reflistings{kaobookcls删除scshape命令}~所示。

\begin{listing}[H]
    \begin{minted}[linenos=false]{latex}
        \addtokomafont{part}{\normalfont\bfseries}
        \addtokomafont{partentry}{\normalfont\bfseries}
    \end{minted}
    \caption{删除 scshape 后的 kaobook.cls‌ 文件}
    \lablistings{kaobookcls删除scshape命令}
\end{listing}

\subsection{kao.sty的变动}
kao.sty文件同样要指定到 styles 文件夹下,修改文件第1行为 \mintinline{latex}|\ProvidesPackage{styles/kao}|。

一些兼容性的修改,如第19行删除了 usenames,改为 \newline \mintinline{latex}|\RequirePackage[dvipsnames,table]{xcolor}|; \newline 第29行增加了 listings=false, 改为 \newline \mintinline{latex}|\AtEndPreamble{\RequirePackage[listings=false]{scrhack}}|; 第240行增加了 singlespacing=true,改为 \newline \mintinline{latex}|\RequirePackage[singlespacing=true]{scrlayer-scrpage}|。

调整了目录的深度,从 section 调整到 subsection,文件第1208行改为 \mintinline{latex}|bookmarksdepth=subsection,|。

\subsection{kaorefs.sty的变动}
kaorefs.sty文件类似地修改到 styles 文件夹下,修改文件第1行为 \mintinline{latex}|\ProvidesPackage{styles/kaorefs}|。

注释掉第41行,以及从48行到148行这段多语言支持的部分。

\section[kao-zh.sty中文支持]{新增kao-zh.sty中文支持文件}

新增的 kao-zh.sty 宏包文件是通过 ctex 包来支持中文的显示,并且通过这个文件对 kaobook 中的关键字展示采用中文重定义,并规定了文档的字体。

\subsection{ctex宏包}

在 kao-zh.sty 文件中引入 ctex 宏包来支持中文,引用时不指定具体字体,后续使用自定义的开源字体代替。

在 \reflistings{kao-zh.sty中引入ctex宏包} 里,如果不准备自定义字体,可以将fontset=none改为对应的mac或windows。

\begin{listing}[H]
    \begin{minted}[linenos=false]{latex}
        \RequirePackage[UTF8, fontset=none, scheme=chinese]{ctex}
    \end{minted}
    \caption{kao-zh.sty中引入ctex宏包}
    \lablistings{kao-zh.sty中引入ctex宏包}
\end{listing}

ctex 是一个专门为支持中文排版而设计的 LaTeX 宏包\sidenote{关于 ctex 宏包的详细文档,可以参考\href{https://www.ctan.org/pkg/ctex}{CTeX 官方文档}},它是LaTeX 中是中文排版里最常用的标准方案,为 LaTeX 提供了对中文字符、字体、段落、编码等的良好支持。

\subsection{标题文本的中文重定义}
对于 kao 样式里定义的一些标题文本,在本项目中进行了重新定义\sidenote{在英文环境中更改特定标签(如目录、插图、表格等)的名称,\textbf{应该有更好的修改方式}。}。如 \reflistings{kao-zh.sty中的标题文本重定义} 里的重新定义命令,可以修改以适配自己需要的中文。

\begin{listing}[H]
    \begin{minted}[linenos=false]{latex}
        \RequirePackage[english]{babel}
        \renewcaptionname{english}{\contentsname}{目录}
        \renewcaptionname{english}{\listfigurename}{插图}
        \renewcaptionname{english}{\listtablename}{表格}
        \renewcaptionname{english}{\indexname}{索引}
        \renewcaptionname{english}{\bibname}{参考文献}
        \renewcaptionname{english}{\figurename}{图}
        \renewcaptionname{english}{\tablename}{表}
        
        \renewcommand{\lstlistlistingname}{代码列表}
    \end{minted}
    \caption{kao-zh.sty中的标题文本重定义}
    \lablistings{kao-zh.sty中的标题文本重定义}
\end{listing}

\subsection{自定义字体}
为了避免使用系统默认字体可能带了的商业使用法律问题,本项目采用了开源免费商用字体。这部分内容可以根据需要自行修改。

\subsubsection{中文字体}

本项目采用 \href{https://github.com/takushun-wu/WenYuanFonts}{WenYuanFonts 项目}提供的文源字体\sidenote{基于思源字体二次开发,开源免费商用。}作为文档字体,另采用 \href{https://github.com/subframe7536/maple-font}{maple-font 项目}提供的Maple Mono 字体\sidenote{开源等宽字体,中英文宽度完美 2:1。}作为文档的等宽字体。所有使用到的字体文件参考 \nrefsec{代码目录结构} 的内容都放在了 font 目录下,且注意,\textbf{字体文件不保证为最新}。

\begin{listing}[H]
    \begin{minted}[linenos=false]{latex}
        \setCJKmainfont [Path={font/}, AutoFakeBold, AutoFakeSlant] {WenYuanMincho-Regular.ttf}
        % 中文无衬线字体
        \setCJKsansfont [Path={font/}, BoldFont={WenYuanGothic-Medium.ttf}, AutoFakeSlant] {WenYuanGothic-Regular.ttf}
        % 中文等宽字体
        \setCJKmonofont [Path={font/}, BoldFont={MapleMonoNormalNL-NF-CN-Bold.ttf},  ItalicFont={MapleMonoNormalNL-NF-CN-Italic.ttf}] {MapleMonoNormalNL-NF-CN-Light.ttf}
    \end{minted}
    \caption{kao-zh.sty中使用的中文字体}
    \lablistings{kao-zh.sty中使用的中文字体}
\end{listing}

如 \reflistings{kao-zh.sty中使用的中文字体} 所示\sidenote{有些地方配置了 AutoFakeBold 和 AutoFakeSlant,没有什么原因,只是懒得去下载字体了。},中文有衬线字体使用文源宋体(WenYuanMincho)、中文无衬线字体使用文源黑体(WenYuanGothic)、等宽字体使用 Maple Mono 字体。

\subsubsection{英文字体}

文档默认的英文字体为 Times New Roman 字体,参考~\nrefsec{Times New Roman字体}~章节内容。可以修改为其他英文字体,曾经比较喜欢的 TeX Gyre 系列字体如~\reflistings{指定英文字体的写法}~所示。

\begin{listing}[H]
    \begin{minted}[linenos=false]{latex}
        \setmainfont{TeX Gyre Termes}
        \setsansfont{TeX Gyre Heros}
        \setmonofont{TeX Gyre Cursor}
        \setmathfont{TeX Gyre Schola Math}
    \end{minted}
    \caption{指定英文字体的写法}
    \lablistings{指定英文字体的写法}
\end{listing}

\iffalse
\section{后续内容}
\subsection{富强民主}
\zhlipsum[3]
\fi