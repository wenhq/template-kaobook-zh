%! Tex program = xelatex

\documentclass[oneside]{styles/kaobok2}
\usepackage{lipsum} % 用于生成示例文本
\usepackage{zhlipsum} % 用于生成示例文本

\begin{document}
%----------------------------------------------------------------------------------
%	BOOK INFORMATION
%----------------------------------------------------------------------------------
\titlehead{
\begin{tikzpicture}[remember picture,overlay]
%%%%%%%%%%%%%%%%%%%% Background %%%%%%%%%%%%%%%%%%%%%%%%
\fill[monOrange] (current page.south west) rectangle (current page.north east);
%%%%%%%%%%%%%%%%%%%% Background Polygon %%%%%%%%%%%%%%%%%%%%
\foreach \i in {2.5,...,22}{
    \node[rounded corners,monOrange!60,draw,regular polygon,regular polygon sides=6, minimum size=\i cm,ultra thick] at ($(current page.west)+(2.5,-5)$) {} ;}
\foreach \i in {0.5,...,22}{
	\node[rounded corners,monOrange!60,draw,regular polygon,regular polygon sides=6, minimum size=\i cm,ultra thick] at ($(current page.north west)+(2.5,0)$) {} ;}
\foreach \i in {0.5,...,22}{
	\node[rounded corners,monOrange!90,draw,regular polygon,regular polygon sides=6, minimum size=\i cm,ultra thick] at ($(current page.north east)+(0,-9.5)$) {} ;}
\foreach \i in {21,...,6}{
	\node[monOrange!85,rounded corners,draw,regular polygon,regular polygon sides=6, minimum size=\i cm,ultra thick] at ($(current page.south east)+(-0.2,-0.45)$) {} ;}
%%%%%%%%%%%%%%%%%%%% Title of the Report %%%%%%%%%%%%%%%%%%%% 
\node[left,white,minimum width=0.625*\paperwidth,minimum height=3cm, rounded corners] at ($(current page.north east)+(0,-9.5)$){{\fontsize{25}{30} \selectfont \bfseries KAOBOOK 学习记录}};
%%%%%%%%%%%%%%%%%%%% Subtitle %%%%%%%%%%%%%%%%%%%% 
\node[left,white,minimum width=0.625*\paperwidth,minimum height=2cm, rounded corners] at ($(current page.north east)+(0,-11)$){{\huge \textit{TEMPLATE LATEX}}};
%%%%%%%%%%%%%%%%%%%% Author Name %%%%%%%%%%%%%%%%%%%% 
\node[left,black!85,minimum width=0.625*\paperwidth,minimum height=2cm, rounded corners] at ($(current page.north east)+(0,-13)$){{\Large \textbf{作者} \textsc{ Wenh\_q}}};
%%%%%%%%%%%%%%%%%%%% Year %%%%%%%%%%%%%%%%%%%% 
\node[rounded corners,text=monOrange,regular polygon,regular polygon sides=6, minimum size=2.5 cm,inner sep=0,ultra thick,fill=monOrange!60] at ($(current page.west)+(2.5,-5)$) {\LARGE \bfseries 2023};
\node[fill=white,ultra thick,draw=monOrange!60,text=monBleu,rounded corners,inner sep=4pt] at ($(current page.south)+(0,2)$) {\bfseries \href{https://github.com}{github.com}} ;
\end{tikzpicture}
}

% \subject{}
\title[我是书籍标题。This is book title. -- \href{https://binwh.com}{binwh.com}]{}

% \subtitle{}
\author[\textit{这是作者}, This is book's author.]{}
\date{}

\frontmatter % START of the pre-document content, uses roman numerals
\if@twoside 
    %----------------------------------------------------------------------------------
%	COPYRIGHT PAGE
%----------------------------------------------------------------------------------
\makeatletter

\newcommand{\uppertext}{
\begin{flushleft}
    \@plaintitle\\ \@plainauthor
\end{flushleft}
}
\newcommand{\lowertext}{
\begin{flushleft}
    \textbf{版权所有} \copyright\ 2024 wenhq测试\\
    \ccbync 此文档采用开源协议\emph{Creative Commons} 《署名—非商业性使用 4.0 协议国际版(CC BY-NC 4.0)》。
    
    如需了解更多信息: \href{https://creativecommons.org/licenses/by-nc/4.0/}{creativecommons.org/licenses/by-nc/4.0/}
    \medskip

    这份文档是使用\href{https://sourceforge.net/projects/koma-script/}{\KOMAScript} 和 \href{ttps://www.latex-project.org/}{\LaTeX},并基于\href{https://github.com/fmarotta/kaobook/}{kaobook}文档类编写的。
    \medskip
    
    \textbf{第1版 --} 2024年12月
    
    \textbf{网络版 --} \href{https://github.com}{github.com}
\end{flushleft}
}
\makeatother

\if@twoside
    \uppertitleback{\uppertext}
    \vspace*{\fill} % 将剩余空间推到顶部
    \lowertitleback{\lowertext}
\else
    \uppertext
    \vspace*{\fill} % 将剩余空间推到顶部
    \lowertext
\fi

    \maketitle
\else 
    \maketitle
    %----------------------------------------------------------------------------------
%	COPYRIGHT PAGE
%----------------------------------------------------------------------------------
\makeatletter

\newcommand{\uppertext}{
\begin{flushleft}
    \@plaintitle\\ \@plainauthor
\end{flushleft}
}
\newcommand{\lowertext}{
\begin{flushleft}
    \textbf{版权所有} \copyright\ 2024 wenhq测试\\
    \ccbync 此文档采用开源协议\emph{Creative Commons} 《署名—非商业性使用 4.0 协议国际版(CC BY-NC 4.0)》。
    
    如需了解更多信息: \href{https://creativecommons.org/licenses/by-nc/4.0/}{creativecommons.org/licenses/by-nc/4.0/}
    \medskip

    这份文档是使用\href{https://sourceforge.net/projects/koma-script/}{\KOMAScript} 和 \href{ttps://www.latex-project.org/}{\LaTeX},并基于\href{https://github.com/fmarotta/kaobook/}{kaobook}文档类编写的。
    \medskip
    
    \textbf{第1版 --} 2024年12月
    
    \textbf{网络版 --} \href{https://github.com}{github.com}
\end{flushleft}
}
\makeatother

\if@twoside
    \uppertitleback{\uppertext}
    \vspace*{\fill} % 将剩余空间推到顶部
    \lowertitleback{\lowertext}
\else
    \uppertext
    \vspace*{\fill} % 将剩余空间推到顶部
    \lowertext
\fi

\fi
\frontmatter
%----------------------------------------------------------------------------------
%	ACKNOWLEDGMENTS PAGE
%----------------------------------------------------------------------------------

\chapter*{\centering 致谢}
\addcontentsline{toc}{chapter}{致谢} % 将致谢添加到目录中
\zhlipsum[23]
\lipsum[23-24]
\zhlipsum[24]
%----------------------------------------------------------------------------------
%	TABLE OF CONTENTS & LIST OF FIGURES/TABLES
%----------------------------------------------------------------------------------

\begingroup % Local scope for the following commands

% Define the style for the TOC, LOF, and LOT
%\setstretch{1} % Uncomment to modify line spacing in the ToC
%\hypersetup{linkcolor=blue} % Uncomment to set the colour of links in the ToC
\setlength{\textheight}{230\hscale} % Manually adjust the height of the ToC pages

% Turn on compatibility mode for the etoc package
\etocclasstocstyle % "toc display" as if etoc was not loaded
\etocstandardlines % "toc lines" as if etoc was not loaded

\tableofcontents % Output the table of contents

\listoffigures % Output the list of figures

% Comment both of the following lines to have the LOF and the LOT on different pages
\let\cleardoublepage\bigskip
\let\clearpage\bigskip

\listoftables % Output the list of tables

\endgroup


\mainmatter % 正文部分开始
\chapter{记录}
在 LaTeX 文档中,\\frontmatter 命令通常用于标记文档的前言部分开始。在 KOMA-Script 文档类(如 scrbook 或 scrreprt)和一些其他文档类中,\\frontmatter 命令之后的内容会使用罗马数字(i, ii, iii, ...)进行页码编号,直到 \\mainmatter 命令,之后的内容会使用阿拉伯数字(1, 2, 3, ...)进行页码编号。

frontmatter 部分通常包括:

封面(Title Page)
版权信息(Copyright Page)
致谢(Acknowledgments)
摘要(Abstract)
目录(Table of Contents)
列表(List of Figures, List of Tables 等)
引言(Introduction)或前言(Preface)
这些部分在书籍和长篇报告中很常见,它们为读者提供了文档的概览和导航。

\chapter{引言chapter}
这是书籍的开始部分。在这里,我们可以介绍书籍的主要内容和结构。

\lipsum[1] % 生成示例文本

\zhlipsum[1]

\section{背景介绍section}
在这个部分,我们将详细介绍书籍的背景信息。

\lipsum[2-3] % 生成示例文本

\zhlipsum[2]

\chapter{章节标题chapter}
这是书籍的第二章节。在这里,我们将深入探讨一些具体的主题。

\lipsum[4-5] % 生成示例文本

\zhlipsum[3]

\section{子章节标题section}
在这个子章节中,我们将讨论一些相关的细节。

\lipsum[6-7] % 生成示例文本

\zhlipsum[4]

\begin{itemize}
    \item zhlipsum[5]
    \item lipsum[8]
    \begin{itemize}
        \item 1234567890
        \begin{itemize}
            \item \zhlipsum[5]
            \item \lipsum[8]
        \end{itemize}
        \item abcd,loO
    \end{itemize}
\end{itemize}

\end{document}

