%! Tex program = xelatex

\documentclass{styles/kaobok2}
\usepackage{lipsum} % 用于生成示例文本
\usepackage{zhlipsum} % 用于生成示例文本

\begin{document}

%----------------------------------------------------------------------------------------
%	BOOK INFORMATION
%----------------------------------------------------------------------------------------
\titlehead{
\begin{tikzpicture}[remember picture,overlay]
%%%%%%%%%%%%%%%%%%%% Background %%%%%%%%%%%%%%%%%%%%%%%%
\fill[monOrange] (current page.south west) rectangle (current page.north east);
%%%%%%%%%%%%%%%%%%%% Background Polygon %%%%%%%%%%%%%%%%%%%%
\foreach \i in {2.5,...,22}{
    \node[rounded corners,monOrange!60,draw,regular polygon,regular polygon sides=6, minimum size=\i cm,ultra thick] at ($(current page.west)+(2.5,-5)$) {} ;}
\foreach \i in {0.5,...,22}{
	\node[rounded corners,monOrange!60,draw,regular polygon,regular polygon sides=6, minimum size=\i cm,ultra thick] at ($(current page.north west)+(2.5,0)$) {} ;}
\foreach \i in {0.5,...,22}{
	\node[rounded corners,monOrange!90,draw,regular polygon,regular polygon sides=6, minimum size=\i cm,ultra thick] at ($(current page.north east)+(0,-9.5)$) {} ;}
\foreach \i in {21,...,6}{
	\node[monOrange!85,rounded corners,draw,regular polygon,regular polygon sides=6, minimum size=\i cm,ultra thick] at ($(current page.south east)+(-0.2,-0.45)$) {} ;}
%%%%%%%%%%%%%%%%%%%% Title of the Report %%%%%%%%%%%%%%%%%%%% 
\node[left,white,minimum width=0.625*\paperwidth,minimum height=3cm, rounded corners] at ($(current page.north east)+(0,-9.5)$){{\fontsize{25}{30} \selectfont \bfseries COURS DE PHYSIQUE}};
%%%%%%%%%%%%%%%%%%%% Subtitle %%%%%%%%%%%%%%%%%%%% 
\node[left,white,minimum width=0.625*\paperwidth,minimum height=2cm, rounded corners] at ($(current page.north east)+(0,-11)$){{\huge \textit{TEMPLATE LATEX}}};
%%%%%%%%%%%%%%%%%%%% Author Name %%%%%%%%%%%%%%%%%%%% 
\node[left,black!85,minimum width=0.625*\paperwidth,minimum height=2cm, rounded corners] at ($(current page.north east)+(0,-13)$){{\Large \textsc{Jimmy Roussel}}};
%%%%%%%%%%%%%%%%%%%% Year %%%%%%%%%%%%%%%%%%%% 
\node[rounded corners,text=monOrange,regular polygon,regular polygon sides=6, minimum size=2.5 cm,inner sep=0,ultra thick,fill=monOrange!60] at ($(current page.west)+(2.5,-5)$) {\LARGE \bfseries 2023};
\node[fill=white,ultra thick,draw=monOrange!60,text=monBleu,rounded corners,inner sep=4pt] at ($(current page.south)+(0,2)$) {\bfseries \href{https://femto-physique.fr/omp/}{femto-physique.fr/omp}} ;
\end{tikzpicture}
}

% \subject{}
\title[Cours sur les outils et méthodes scientifiques -- \href{https://femto-physique.fr}{femto-physique.fr}]{}

% \subtitle{}
\author[\textsc{Jimmy Roussel}, professeur agrégé à l'Ecole Nationale Supérieure de Chimie de Rennes]{}
\date{}

%----------------------------------------------------------------------------------------
% START of the pre-document content, uses roman numerals
% ---------------------------------------------------------------------------------------
\frontmatter 
%----------------------------------------------------------------------------------------
%	COPYRIGHT PAGE
%----------------------------------------------------------------------------------------
\makeatletter
\uppertitleback{\@plaintitle\\ \@plainauthor} % Header
\lowertitleback{
\textbf{Copyright} \copyright\ 2023 Jimmy Roussel\\
\ccbync\ Ce document est sous licence \emph{Creative Commons} «Attribution - Pas d’Utilisation Commerciale 4.0 International (CC BY-NC 4.0)».

Pour plus d'informations: \href{https://creativecommons.org/licenses/by-nc/4.0/}{creativecommons.org/licenses/by-nc/4.0/}
\medskip

Ce document est réalisé avec l'aide de \href{https://sourceforge.net/projects/koma-script/}{\KOMAScript} et  \href{ttps://www.latex-project.org/}{\LaTeX} en utilisant la classe \href{https://github.com/fmarotta/kaobook/}{kaobook}.
\medskip

\textbf{1 édition --} Janv. 2011

\textbf{Version en ligne --} \href{https://femto-physique.fr/omp/}{femto-physique.fr/omp}
}
\makeatother

%----------------------------------------------------------------------------------------
%	OUTPUT TITLE PAGE AND PREVIOUS
%----------------------------------------------------------------------------------------
% Note that \maketitle outputs the pages before here
\maketitle

%----------------------------------------------------------------------------------------
%	TABLE OF CONTENTS & LIST OF FIGURES/TABLES
%----------------------------------------------------------------------------------------

\begingroup % Local scope for the following commands

% Define the style for the TOC, LOF, and LOT
%\setstretch{1} % Uncomment to modify line spacing in the ToC
%\hypersetup{linkcolor=blue} % Uncomment to set the colour of links in the ToC
\setlength{\textheight}{230\hscale} % Manually adjust the height of the ToC pages

% Turn on compatibility mode for the etoc package
\etocclasstocstyle % "toc display" as if etoc was not loaded
\etocstandardlines % "toc lines" as if etoc was not loaded

\tableofcontents % Output the table of contents

\listoffigures % Output the list of figures

% Comment both of the following lines to have the LOF and the LOT on different pages
\let\cleardoublepage\bigskip
\let\clearpage\bigskip

\listoftables % Output the list of tables

\endgroup

\chapter{引言chapter}
这是书籍的开始部分。在这里,我们可以介绍书籍的主要内容和结构。

\lipsum[1] % 生成示例文本

\zhlipsum[1]

\section{背景介绍section}
在这个部分,我们将详细介绍书籍的背景信息。

\lipsum[2-3] % 生成示例文本

\zhlipsum[2]

\chapter{章节标题chapter}
这是书籍的第二章节。在这里,我们将深入探讨一些具体的主题。

\lipsum[4-5] % 生成示例文本

\zhlipsum[3]

\section{子章节标题section}
在这个子章节中,我们将讨论一些相关的细节。

\lipsum[6-7] % 生成示例文本

\zhlipsum[4]

\begin{itemize}
    \item zhlipsum[5]
    \item lipsum[8]
    \begin{itemize}
        \item 1234567890
        \begin{itemize}
            \item \zhlipsum[5]
            \item \lipsum[8]
        \end{itemize}
        \item abcd,loO
    \end{itemize}
\end{itemize}

\end{document}

