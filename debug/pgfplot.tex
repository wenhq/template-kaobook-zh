\documentclass[border=5pt]{standalone} % standalone类适合单独编译图表
\usepackage[UTF8]{ctex}  % 中文支持
\usepackage{pgfplots}    % 加载pgfplots
\pgfplotsset{compat=newest}% 使用兼容模式(根据你的版本调整)

% 设置中文字体(如果需要特定字体)
% \setCJKmainfont{SimSun}

% 定义图表全局样式
\pgfplotsset{
    every axis/.append style={
        font=\small,          % 全局字体大小
        line width=0.5pt,     % 轴线宽度
        tick style={color=black}, % 刻度样式
    },
    every axis label/.append style={font=\small}, % 轴标签样式
    every tick label/.append style={font=\footnotesize}, % 刻度标签样式
    every legend/.append style={font=\footnotesize}, % 图例样式
}

\begin{document}

\begin{tikzpicture}
    \begin{axis}[
        width=10cm,                      % 图表宽度
        height=7cm,                      % 图表高度
        xlabel={横坐标 ($x$)},           % 中文横轴标签
        ylabel={纵坐标 ($y$)},           % 中文纵轴标签
        title={测试数据可视化},           % 中文标题
        grid=both,                       % 显示网格
        legend pos=north west,           % 图例位置
        legend cell align=left,          % 图例对齐方式
    ]
    
    % 第一条曲线
    \addplot[color=red, mark=square, thick] 
        table[x=x, y=y1, col sep=comma] {data/test.csv};
    \addlegendentry{数据集一}
    
    % 第二条曲线
    \addplot[color=blue, mark=triangle, thick] 
        table[x=x, y=y2, col sep=comma] {data/test.csv};
    \addlegendentry{数据集二}
    
    % 第三条曲线
    \addplot[color=green!60!black, mark=o, thick] 
        table[x=x, y=y3, col sep=comma] {data/test.csv};
    \addlegendentry{数据集三}
    
    % 添加注释文本
    \node[anchor=south east, align=left, font=\small] 
        at (rel axis cs:0.98,0.02) {注:这是一个\\测试图表};
    
    \end{axis}
\end{tikzpicture}

\end{document}
